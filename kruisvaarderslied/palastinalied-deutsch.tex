\documentclass[11pt]{article}

\usepackage{multicol}
\usepackage{verse}
\usepackage{longtable}
%\usepackage{tabls}
\usepackage[a4paper,margin=1cm]{geometry}

\begin{document}

\newlength{\vcolwidth}
\setlength{\vcolwidth}{0.3\textwidth}

\newlength{\ssep}
\setlength{\ssep}{3ex}

%\tablinesep=3ex
%\tabcolsep=10pt

\newenvironment{stanza}{
}{
}
\begin{verse}
\poemtitle{Pal\"astinalied}

\begin{multicols}{2}

%
% First stanza
%
\begin{stanza}
N\^u lebe ich mir alr\^erst werde, \\
s\^\i t m\^\i n s\"undic ouge sihet \\
daz h\^ere lant und ouch die erde, \\
der man vil der \^eren gihet.\\
N\^u ist geschehen, des ich ie bat:\\
ich bin komen an die stat,\\
d\^a got mennischl\^\i chen trat.\\!
\end{stanza}


%
% Second stanza
%
\begin{stanza}
Sch\oe niu lant r\^\i ch unde h\^ere, \\
swaz ich der noch h\^an gesehen, \\
s\^o bist d\^uz ir aller \^ere. \\
Waz ist wunders hie geschehen! \\
Daz ein maget ein kint gebar, \\
h\^ere \"uber aller engel schar, \\
was daz niht ein wunder gar?\\!
\end{stanza}
%
% Third stanza
%
\begin{stanza}
Hie liez er sich reine toufen, \\
daz der mensche reine s\^\i. \\
D\^o liez er sich hie verkoufen, \\
daz wir eigen wurden fr\^\i. \\
Anders w\ae ren wir verlorn. \\
Wol dir, sper, kriuze unde dorn! \\
W\^e dir, heiden, daz ist dir zorn!\\!
\end{stanza}
%
% Fourth stanza
%
\begin{stanza}
D\^o er sich wolte \"ubr uns erbarmen, \\
hie leit er den grimmen t\^ot, \\
er vil r\^\i che durch uns armen, \\
daz wir k\oe men \^uz der n\^ot. \\
Daz in d\^o des niht verdr\^oz, \\
dast ein wunder alze gr\^oz, \\
aller wunder \"ubergen\^oz. \\!
\end{stanza} 
%
% Fifth stanza
%
\begin{stanza}
Hinnen fuor der sun zer helle\\
von dem grabe, d\^a er inne lac.\\
Des was ie der vater geselle,\\
und der geist, den nieman mac\\
sunder scheiden: \^est al ein,\\
sleht und ebener danne ein zein,\\
als er Abrah\^ame erschein.\\!
\end{stanza} 
%
% Sixth stanza
%
\begin{stanza}
D\^o er den tievel d\^o geschande,\\
daz nie keiser baz gestreit,\\
d\^o fuor er her wider ze lande.\\
D\^o huob sich der juden leit,\\
daz er herre ir huote brach,\\
und daz man in s\^\i t lebendic sach,\\
den ir hant sluoc unde stach.\\!
\end{stanza} 
%
% Seventh stanza
%
\begin{stanza}
Dar n\^ach was er in dem lande\\
vierzic tage: d\^o fuor er dar,\\
dannen in s\^\i n vater sande.\\
S\^\i nen geist, der uns bewar,\\
den sante er hin wider zehant.\\
Heilic ist daz selbe lant:\\
s\^\i n name, der ist vor gote erkant.\\!
\end{stanza} 
%
% Eighth stanza
%
\begin{stanza}
In diz lant h\^at er gesprochen\\
einen angesl\^\i chen tac,\\
d\^a diu witwe wirt gerochen\\
und der weise klagen mac\\
und der arme den gewalt,\\
der d\^a wirt an ime gestalt.\\
Wol ime dort, der hie vergalt!\\!
\end{stanza} 
%
% Ninth stanza
%
\begin{stanza}
Unser lantreht\ae re tihten\\
fristet d\^a niemannes klage;\\
wan er wil d\^a zestunt rihten,\\
s\^o ez ist an dem lesten tage:\\
swer deheine schulde hie l\^at\\
unverebenet, wie der st\^at\\
dort, d\^a er pfant noch b\"urgen h\^at!\\!
\end{stanza} 
%
% Tenth stanza
%
\begin{stanza}
Kristen, juden unde heiden\\
jehent, daz diz ir erbe s\^\i:\\
got m\"ueze ez ze rehte scheiden\\
durch die s\^\i ne namen dr\^\i.\\
Al diu werlt diu str\^\i tet her.\\
Wir s\^\i n an der rehten ger:\\
reht ist, daz er uns gewer.\\!
\end{stanza} 
%
% Eleventh stanza
%
\begin{stanza}
N\^u l\^at iuch des niht verdriezen,\\
daz ich noch gesprochen h\^an.\\
Ich wil iu die rede entsliezen\\
kurzl\^\i ch und iuch wizzen l\^an,\\
swaz got mit dem menschen ie\\
wunders in der werlt begie,\\
daz huop sich und endet hie.\\!
\end{stanza} 

\end{multicols}

\end{verse}%
\end{document}
